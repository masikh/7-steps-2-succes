\documentclass[8pt]{extarticle}
\usepackage{lipsum}
\usepackage{enumitem}
\usepackage{nopageno}
\usepackage[dutch]{babel}

\title{7 Steps 2 Success}
\author{Robert Nagtegaal}
\date{24 juni 2024}

\begin{document}

\maketitle


\section*{Inleiding}
\noindent Veel projecten ontstaan vanuit een bepaalde wens of behoefte. Er is een terugkerende “situatie” waarin maatwerk moet worden geleverd om deze behoefde te vervullen. Omdat maatwerk geen standaard kent is de kans groot dat deze “situatie” keer op keer een net andere aanpak krijgt wat kan leiden tot problemen. De wens is om structureel deze problemen op te lossen waarmee het project is geboren.\\
\\
Het aanpakken van een nieuw project vereist een gestructureerde en methodische benadering. Dit zorgt ervoor dat alle aspecten van de wens of behoefte goed worden begrepen. De terugkerende “situatie” kan door deze methodische aanpak goed worden beheerd.\\
\\
Het is hierbij belangrijk om op te merken dat de wensen door de tijd veranderen, bijvoorbeeld door “voortschrijdend inzicht” of een verandering in de “situatie”. Een “agile” aanpak kan dan zeer gewenst zijn om dit veranderende landschap het hoofd te bieden.\\
\\
Hieronder probeer ik middels zeven stappen een rode draad uit te zetten waarlangs een software-project kan worden begeleid.

\newpage

%
% STAP 1
%

\section*{Stap 1: Verkennen van de omgeving}
\textbf{Doel:} Begrijp de context van het project, inclusief de omgeving (het bedrijf), de stakeholders, de bestaande processen (systemen), en de doelstellingen.

\begin{itemize}
    \item \textbf{Bedrijfscultuur en doelstellingen:} Wat is de visie van het bedrijf of organisatie en welke waarden draagt het uit. Hoe past het project binnen de strategische doelstellingen van deze organisatie.
    \item \textbf{Stakeholders identificeren:} Wie zijn de stakeholders van het project. Dit kunnen interne- of externe gebruikers zijn, leveranciers, etc.  Organiseer gesprekken en interviews om de behoefde van deze stakeholders te begrijpen.
    \item \textbf{Huidige systemen en processen:} Welke bestaande processen, systemen of architectuur hebben invloed op het project. 
    \item \textbf{Data types en integratie:} Identificeer welke data er voorhanden is en hoe deze in het nieuwe systeem worden geïntegreerd.
\end{itemize}

%
% STAP 2
%


\section*{Stap 2: Definiëren van de projectdoelstellingen en scope}
\textbf{Doel:} Vaststellen wat het project moet opleveren en wat de grenzen zijn.

\begin{itemize}
    \item \textbf{Projectdoelstellingen formuleren:} Definieer de specifieke doelen van het project (bijv. verbetering van prestaties, nieuwe functionaliteiten, etc.).
    \item \textbf{Scope bepalen:} Stel de grenzen van het project vast. Wat wordt wel en niet opgeleverd? Welke functies en features zijn essentieel?
    \item \textbf{Succescriteria:} Bepaal hoe succes gemeten kan worden (bijv. KPI’s (S.M.A.R.T.), gebruikersacceptatie, prestaties).
\end{itemize}

%
% STAP 3
%

\section*{Stap 3: Opstellen van een projectplan}
\textbf{Doel:} Creëer een gedetailleerde blauwdruk voor het project.

\begin{itemize}
    \item \textbf{Tijdlijn en milestones:} Maak een project-tijdlijn met belangrijke mijlpalen en deadlines.
    \item \textbf{Taakverdeling:} Wijs taken toe aan teamleden, gebaseerd op hun vaardigheden en expertise.
    \item \textbf{Risicoanalyse en mitigatie:} Identificeer potentiële risico’s en stel een plan op om deze te mitigeren.
\end{itemize}

%
% STAP 4
%

\section*{Stap 4: Ontwerp van de software architectuur}
\textbf{Doel:} Bepaal de technische fundamenten van het systeem.

\begin{itemize}
    \item \textbf{Systeemarchitectuur:} Ontwerp de high-level architectuur van het systeem, inclusief componenten en hun interacties.
    \item \textbf{Technologie-stack:} Kies de technologieën, tools en platforms die gebruikt zullen worden.
    \item \textbf{Data architectuur:} Definieer hoe data wordt opgeslagen, beheerd en verwerkt.
\end{itemize}

%
% STAP 5
%

\section*{Stap 5: Implementatie}
\textbf{Doel:} Bouw de software volgens het ontwerp en de specificaties.

\begin{itemize}
    \item \textbf{Ontwikkeling:} Schrijf de code en bouw de verschillende componenten van het systeem.
    \item \textbf{Integratie:} Zorg voor integratie met bestaande systemen en datastromen.
    \item \textbf{Testing:} Voer unit tests, integratietests en systeemtests uit om bugs en problemen te identificeren en op te lossen.
\end{itemize}

%
% STAP 6
%

\section*{Stap 6: Deployment en training}
\textbf{Doel:} Zet het systeem in productie en zorg dat gebruikers er effectief mee kunnen werken.

\begin{itemize}
    \item \textbf{Deployment planning:} Bereid de productieomgeving voor en voer de daadwerkelijke implementatie uit.
    \item \textbf{Gebruikerstraining:} Train de eindgebruikers zodat ze het systeem effectief kunnen gebruiken.
    \item \textbf{Documentatie:} Maak duidelijke documentatie voor gebruikers en beheerders.
\end{itemize}

%
% STAP 7
%

\section*{Stap 7: Evaluatie en onderhoud}
\textbf{Doel:} Zorg voor continue verbetering en ondersteuning van het systeem.

\begin{itemize}
    \item \textbf{Post-implementatie evaluatie:} Evalueer het project na en tijdens de implementatie. Analyseer wat goed gaat en wat beter kan.
    \item \textbf{Onderhoud en ondersteuning:} Zet een plan op voor regelmatige updates, bug-fixes en ondersteuning.
    \item \textbf{Feedback-loop:} Verzamel feedback van gebruikers en stakeholders om (toekomstige) verbeteringen te identificeren.
\end{itemize}

\section*{Agile aanpak}
Het nu lijkt dat het project na stap 7 klaar is maar zoals in de inleiding al is besproken veranderen de wensen door “voortschrijdend inzicht” of een verandering in de “situatie”. Het is dan ook logisch dat alle stappen regelmatig opnieuw worden bezocht en tegen het licht gehouden. Zeker stap 4 tot en met 7 zullen regelmatig de revue passeren.
\end{document}

